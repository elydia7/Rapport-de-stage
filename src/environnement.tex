\begin{center}
\section{\colorbox{orange}{Environnement Professionnel}}
\end{center}


\subsection{Histoire de Secuserve}
Le chef d'entreprise Monsieur Stéphane Bouche la créée en 2003 suite à la montée des cybers attaques via les comptes de messagerie dans les entreprises et administrations. Fort de son expérience, Secuserve propose donc des services de protection des emails entrant et sortant à toutes les entreprises quel que soit leur domaine d'activité ou leur taille. Si cela résume ce qu'est l'offre Secuserve, celle-ci se décline en trois produits ou solutions distincte.

\subsection{Organisation de l'entreprise}
Comme évoqué précédemment, cette entreprise est une petite structure dont le siège social est à Paris mais qui possède des bureaux à Lyon (Villeurbanne pour être précis). 
Secuserve évolue sur le marché global de l’informatique mais est spécialisée sur le segment de la sécurité lié à la messagerie électronique.
\subsection{Organigramme Secuserve}
Concrètement, la structure est répartie sur les deux zones (Paris et Villeurbanne).
Les équipes de Paris sont dédiées au commerce avec trois commerciaux dont la directrice commerciale Mme Aline Breton. Le directeur technique est également installé dans la capitale où sont les datacenters de la société pour une question de proximité en cas de problème. Le site de Lyon est dédié à la recherche et développement dont l’activité est reconnue par le ministère de l’enseignement et de la recherche. J’ai été recruté pour développer la partie commerciale sur le sud de la France avec Monsieur Edouard Courbon, ingénieur d’affaires,  qui a quitté la société en Février 2017 pour un post aux Etats-Unis. De fait, nous serons deux sur cette partie à compter de fin février 2016 : le PDG et moi.
Tout ce qui appartient au domaine administratif et comptable est géré par Mme Gaillac à Paris. (voir Annexe5 page.\pageref{organigramme})

\subsection{L’offre Secuserve}
Secuserve propose trois services, qui sont les suivants:
\subsubsection{E-securemail}
Il s’agit d’un service de filtrage (antispam et antivirus) et de sécurisation des emails.
\subsubsection{Optimails:} C’est un service de messagerie d’entreprise collaborative et externalisée.
\subsubsection{Sharecan:}Service de stockage et de partage de fichiers en ligne.
Il s’agit d’un marché très spécifique mais sur lequel la concurrence est accrue.

\subsection{concurrence:}
Il s’agit d’un marché très spécifique mais sur lequel la concurrence est accrue.\\

%tableau centré à taille variable qui s'ajuste automatiquement suivant la longueur du contenu
\begin{figure}[!h]
\begin{center}
\begin{tabular}{|l|l|l|l|c|l|l|}
  \hline
 Entreprise & Chiffre d 'affaire 2015& Parts marché\\
  \hline
   Vade Secure& 4 227 220 euros & 28.49\% \\
   \hline
  Mail in Black& 2 695 500 euros & 18.17\% \\
  \hline
  AltoSpam& 1 038 700euros & 7.00\% \\
  \hline  
   \rowcolor{orange}Secuserve& 873 700 euros& 5.85\% \\
	\hline
   Autres(microsoft,symentec,etc...)& 6 000 000 euros & 40\%\\
	\hline  
    Total& 14 835 100 & 100\% \\
  \hline
\end{tabular}
\end{center}
\caption{Parts de marché du secteur}
\end{figure}

