\begin{center}
%retour sur l'experience
\section{\colorbox{orange}{Retour sur l'expérience}}
\end{center}

\subsection{Compétences acquises}
Ce stage m'a permis d'apprendre et de maitriser différents outils dans la technologie du Web. J'ai appris comment créer des interfaces administrable avec (php, mysql, jquery, etc..) et en adoptant la même méthode de travail que Responsable R\&D avec lequel j'ai travaillé sur le projet. Et au cours du stage, j'ai eu le réflexe de garder une homogénéité au niveau du code c'est-à dire que j'appelle les noms de variables ou les noms de fonctions de la même manière que mon tuteur les appellent, à présent je maitrise ces outils. Je sais comment les utiliser et je peux désormais m'en servir pour mes futurs projets. Et enfin, j'ai la capacité de distinguer et de détecter des erreurs qui sont sont liées à ces outils car on à des bugs de temps en temps qu'on fini toujours par corriger. 
\subsection{Méthodes de travail}
J'ai découvert des méthodes de travail et des moyens de communications pendant ma période de stage. Le moyen de communication utilisé au sein de l'équipe SECUSERVE  était l'application internet qui s'appelle PRONTO. C'est une interface de messageries qu'on utilise entre équipes dans laquelle on a des notifications sur tout les commits fait par les développeurs et pour s'échanger divers types de données.
Un autre outil qu'on utilise pour s'organiser et pour avoir clairement les taches qu'on doit faire est cette ce logiciel XMIND sur lequel on a en même temps défini toutes les pages qu'on à développer mais aussi des drapeaux sur les taches qui permettent de connaitre l'évolution et l'état du projet. Cet outils m'a permis de m'organiser et avoir les idées claires des tâches qu'on doit faire. Donc cet outil est une très bonne méthode de travail que j'ai adapté et que je vais pouvoir utiliser dans le futur.  

\subsection{Intégration dans l’environnement}
Au début du stage on à installé les matériels avec le personnel de l'entreprise les employés et les stagiaires sa nous à faciliter l'intégration dans l'équipe.
Dans cette agence je travaille avec le responsable R\&D qui m'explique les tâches à faire le matin a 9h puis on fait un point vers 11h et l'après-midi également, avec lequel j'échange beaucoup informations notamment sur des sujet en informatique et de technique de travail.
Entre général 12h et 14h on mange tous ensemble et sa nous permet d'avoir quelque retours de travail des fois aussi avec les autres stagiaire qui travail sur le gestionnaire gitLab lequel on doit on doit utiliser prochainement.

Ce stage demande de faire preuve d'autonomie. Pour ne pas trop déranger responsable de développement qui travail aussi sur ces taches. Donc j'ai appris à utiliser certains outils sur internet notamment sur les sites (php.net, www.w3schools.com, getbootstrap.com) et des mes connaissance acquises en licence. 
